 
\section{První sekce}
\begin{frame}{Nadpis prvního slajdu}{Podnadpis}
Tady něco napíšu.
\end{frame}

\begin{frame}{Nadpis druhého slajdu}
Dovnitř můžu dát třeba seznam. Beamer je:

\begin{itemize}
\item jednoduchý, \pause
\item čistý, \pause
\item rychlý. \pause
\end{itemize}

\begin{itemize}[<+->]
\item jednoduchý,
\item čistý,
\item rychlý.
\end{itemize}

\end{frame}

\section{Bloky}
\begin{frame}{Bloky}

\pause
\begin{exampleblock}{Příklad}
\begin{enumerate}
\item jedna
\item dva
\end{enumerate}
\end{exampleblock}
\pause
\begin{block}{Obecný blok}
Uvnitř bloku nemusí být seznam.
\end{block}
\pause
\begin{alertblock}{Pozor!}
Barvy bloků určujeme stylem. V kódu je používáme sémanticky, tenhle je výstražný!
\end{alertblock}
\end{frame}


\section{Závěr}
\begin{frame}[fragile]
Slajd nemusí mít nadpis a může být rozdělen do sloupců.
\begin{columns}
\begin{column}{.4\textwidth}
\begin{exampleblock}{Ahoj světe}
\end{exampleblock}
\end{column}
\begin{column}{.4\textwidth}
\begin{center}
\includegraphics[width=.9\textwidth]{logo-cvut}
% obrázky, tabulky v prezentaci bez plovoucího prostředí
\end{center}
\end{column}
\end{columns}
\end{frame}

\begin{frame}{Shrnutí}
\begin{itemize}
\item  Cíl práce byl to udělat.
\item Udělal jsem to.
\item Dalo mi to hodně práce.
\item Výsledek je hooodně dobrý.
\end{itemize}

\end{frame}